\documentclass{report}

%%%%%%%%%%%%%%%%%%%%%%%%%%%%%%%%%%%%%%%%%
% Wenneker Assignment
% Structure Specification File
% Version 2.0 (12/1/2019)
%
% This template originates from:
% http://www.LaTeXTemplates.com
%
% Authors:
% Vel (vel@LaTeXTemplates.com)
% Frits Wenneker
%
% License:
% CC BY-NC-SA 3.0 (http://creativecommons.org/licenses/by-nc-sa/3.0/)
% 
%%%%%%%%%%%%%%%%%%%%%%%%%%%%%%%%%%%%%%%%%

%----------------------------------------------------------------------------------------
%	PACKAGES AND OTHER DOCUMENT CONFIGURATIONS
%----------------------------------------------------------------------------------------

\usepackage{amsmath, amsfonts, amsthm} % Math packages

\usepackage{graphicx} % Required for inserting images
\graphicspath{{Figures/}{./}} % Specifies where to look for included images (trailing slash required)

\usepackage{booktabs} % Required for better horizontal rules in tables

\usepackage{dirtytalk} % Required for quoting.

\usepackage{float} % Added for hard placement of images.
\usepackage{subfig}
\usepackage{caption}

\usepackage[dvipsnames]{xcolor} % Added for extra colors.

\usepackage{tikz} % For colored boxes and more.

\usepackage{hyperref}
\hypersetup{
    colorlinks,
    linkcolor={black!50!black},
    citecolor={blue!50!black},
    urlcolor={blue!80!black}
} % prettier links
\usepackage{url} % For URLs

\numberwithin{equation}{section} % Number equations within sections (i.e. 1.1, 1.2, 2.1, 2.2 instead of 1, 2, 3, 4)
\numberwithin{figure}{section} % Number figures within sections (i.e. 1.1, 1.2, 2.1, 2.2 instead of 1, 2, 3, 4)
\numberwithin{table}{section} % Number tables within sections (i.e. 1.1, 1.2, 2.1, 2.2 instead of 1, 2, 3, 4)

\usepackage{enumitem} % Required for list customisation
\setlist{noitemsep} % No spacing between list items

%----------------------------------------------------------------------------------------
%	DOCUMENT MARGINS
%----------------------------------------------------------------------------------------

\usepackage{geometry} % Required for adjusting page dimensions and margins

\geometry{
	paper=a4paper, % Paper size, change to letterpaper for US letter size
	top=2.5cm, % Top margin
	bottom=3cm, % Bottom margin
	left=3cm, % Left margin
	right=3cm, % Right margin
	headheight=0.75cm, % Header height
	footskip=1.5cm, % Space from the bottom margin to the baseline of the footer
	headsep=0.75cm, % Space from the top margin to the baseline of the header
	% showframe, % Uncomment to show how the type block is set on the page
}

%----------------------------------------------------------------------------------------
%  CODE LISTINGS
%----------------------------------------------------------------------------------------
\usepackage{minted} % Required for insertion of code
\newminted{python}{
    frame=lines, 
    framesep=2mm, 
    baselinestretch=1.2, 
    fontsize=\footnotesize, 
    linenos, 
    breaklines, 
    autogobble,
    tabsize=4,
    bgcolor=black!5
} % shortcut for minter environment. Use with \begin{pythoncode} ... \end{pythoncode}

%----------------------------------------------------------------------------------------
%	FONTS & LANGUAGE
%----------------------------------------------------------------------------------------

\usepackage{fontspec, polyglossia, unicode-math} % Required for specifying custom fonts under LuaLaTeX

\setmainlanguage{greek}
\setotherlanguage{english}


\setmainfont{GFS Didot} % Main document font
\setsansfont{GFS Didot}
\setmonofont{mononoki} % Font for code listings

\newfontfamily\greekfont[Script=Greek]{GFS Didot}
\newfontfamily\greekfontsf[Script=Greek]{GFS Didot}


\newcommand{\en}[1]{\foreignlanguage{english}{#1}}
\newcommand{\gr}[1]{\foreignlanguage{greek}{#1}} % won't need these

%----------------------------------------------------------------------------------------
%	SECTION TITLES
%----------------------------------------------------------------------------------------

\usepackage{sectsty} % Allows customising section commands

\sectionfont{\vspace{6pt}\centering\normalfont\scshape} % \section{} styling
\subsectionfont{\normalfont\bfseries} % \subsection{} styling
\subsubsectionfont{\normalfont\itshape} % \subsubsection{} styling
\paragraphfont{\normalfont\scshape} % \paragraph{} styling

%----------------------------------------------------------------------------------------
%	HEADERS AND FOOTERS
%----------------------------------------------------------------------------------------

\usepackage{scrlayer-scrpage} % Required for customising headers and footers

\ohead*{} % Right header
\ihead*{} % Left header
\chead*{} % Centre header

\ofoot*{} % Right footer
\ifoot*{} % Left footer
\cfoot*{\pagemark} % Centre footer

%----------------------------------------------------------------------------------------
%	CUSTOM COMMANDS
%----------------------------------------------------------------------------------------

% Define a command for inserting images [caption]{image}{size}
\newcommand{\img}[3][htbp]{%
    \begin{figure}[H]
        \centering
        \fcolorbox{black}{white}{\includegraphics[height=#3em]{#2}}
        \ifx&#1&%
        \else
            \caption{#1}
        \fi
    \end{figure}
}

% Two images side by side {image1}{caption1}{image2}{caption2
\newcommand{\twoimgs}[4]{%
    \begin{figure}[H]
        \centering
        \begin{minipage}[b]{0.45\textwidth}
            \centering
            \fcolorbox{black}{white}{\includegraphics[width=0.95\textwidth]{#1}}
            \caption{#2}
        \end{minipage}
        \hfill
        \begin{minipage}[b]{0.45\textwidth}
          \centering
            \fcolorbox{black}{white}{\includegraphics[width=0.95\textwidth]{#3}}
            \caption{#4}
        \end{minipage}
    \end{figure}
}

% Better chapter formatting
\makeatletter
\def\@makechapterhead#1{%
  \vspace*{50\p@}%
  {\parindent \z@ \raggedright \normalfont
    \ifnum \c@secnumdepth >\m@ne
      %\if@mainmatter
        %\huge\bfseries \@chapapp\space \thechapter
        \huge\bfseries \thechapter.\space%
        %\par\nobreak
        %\vskip 20\p@
      %\fi
    \fi
    \interlinepenalty\@M
    \huge \bfseries #1\par\nobreak
    \vskip 40\p@
  }}
\makeatother


% Extra Formatting
\setlength{\parindent}{0em}
\setlength{\parskip}{0em}
 % Include the file specifying the document structure and custom commands

\begin{document}

\chapter{Part A}
\section{Loading a Watertight Mesh}

Ως watertight mesh, η Open3D ορίζει ένα Mesh το οποίο είναι edge manifold, vertex manifold και όχι self intersecting. Πιο απλά,
ένα watertight mesh είναι ένα mesh που αποτελείται από μία κλειστή επιφάνεια, χωρίς κενά, με ξεκάθαρο μέσα και έξω. 
\\\\
Τα περισσότερα meshes που είναι διαθέσιμα online είναι watertight. Στα πλαίσια τις εργασίας χρησιμοποιήθηκε το μοντέλο ενός αγάλματος γάτας
\cite{concrete-cat-statue}. Περιλαμβάνει 24.4k vertex και 12.0k faces και είναι watertight.
Τα textures του μοντέλου αυτού δεν χρησιμοποιήθηκαν, καθώς δεν είναι απαραίτητα για την εργασία.

% \img[Το μοντέλο της γάτας που χρησιμοποιήθηκε]{images/cat-model.png}{20}

\section{Porjection of model onto a plane}
\subsection{Δημιουργία του επιπέδου}

Για να προβληθεί το μοντέλο της γάτας στο επίπεδο, πρέπει πρώτα να δημιουργηθεί το επίπεδο. Βρίσκοντας το bounding box του μοντέλου
και προσθέτοντας ένα μικρό offset στο z, δημιουργείται το επίπεδο. Το επίπεδο αυτό είναι παράλληλο στο επίπεδο x-y και βρίσκεται πίσω
από το μοντέλο. Για να μπορούν να εμφανιστούν ενδιαφέροντες προβολές, υπάρχει η δυνατότητα περιστροφής του επιπέδου γύρω από τον άξονα y.
\\\\
Για τους υπολογισμούς που θα γίνουν είναι βοηθητικό να γίνουν στο σύστημα συντεταγμένων του επιπέδου αντί του μοντέλου.
Για τον λόγο αυτό υπολογίζεται ένα rotation matrix που μεταφέρει το σύστημα συντεταγμένων του μοντέλου στο σύστημα συντεταγμένων του επιπέδου.
Αυτο το rotation matrix υπολογίζεται με βάση το κανονικό διάνυσμα του επιπέδου. Το διάνυσμα πρέπει να περισταφεί ώστε να
πέφτει πάνω στο $\begin{bmatrix}0 \\ 0 \\ 1\end{bmatrix}$. Εδώ χρησιμοποιήθηκε μία τροποποιημένη μορφή του Rodrigues' rotation formula
\cite{rodrigues-rotation-formula}
που περιλαμβάνει matrices και υπολογίζει το matrix $R$ που περιστρέφει το διάνυσμα $a$ στο διάνυσμα $b$. Δηλαδή $Ra = b$. Μαζί με αυτό
υπολογίζεται και το αντίστροφό του $R^{-1}$ για την αντίθετη μετατροπή.

% \img[Το επίπεδο προβολής πίσω από το μοντέλο]{images/projection-plane.png}{20}

\subsection{Δειγματοληψία του επιπέδου}
Για την ομοιόμορφη δειγματοληψία του επιπέδου χρησιμοποιήθηκε η συνάρτηση create\_random της κλάσσης PointCloud από
την δοθέντη βιβλιοθήκη vvrpywork. Η συνάρτηση όμως, δέχεται σαν όρια για την δειγματοληψία ένα Cuboid. Αν σαν ορίσματα έπαιρνε 2 αντίθετες
κορυφές του επιπέδου, τότε θα επέστρεφε σημεία μέσα στο ορθογώνιο που ορίζουν και όχι μόνο στην επιφάνειά του (όπως απαιτείται).
\\\\
Εδώ χρειάζεται η αλλαγή του συστήματος συντεταγμένων. Στο σύστημα συντεταγμένων του επιπέδου, το ορθογώνιο που ορίζεται από 2 αντίθετες
κορυφές του ταυτίζεται με το επίπεδο καθώς τα 2 σημεία έχουν τις ίδιες z συντεταγμένες. Έτσι, δημιουργούνται τα σημεία στο σύστημα συντεταγμένων
του επιπέδου και μετατρέπονται έπειτα στο σύστημα συντεταγμένων του μοντέλου για να εμφανιστούν στα σωστά σημεία.
\\\\
Για αυτά τα σημεία, γίνεται ένας έλεγχος αν ανήκουν στην προβολή του επιπέδου ή όχι. Όσα βρίσκονται εντός της προβολής χρωματίζονται με
κόκκινο και τα υπόλοιπα με μαύρο. Στο τέλος, εμφανίζεται με κόκκινο μία προσέγγιση της προβολής του μοντέλου στο επίπεδο. Όσα περισσότερα
σημεία, τόσο καλύτερη η προσέγγιση.

\img[Προσέγγιση της προβολής του μοντέλου στο επίπεδο]{images/projection-point-approximation.png}{20}

\subsection{Υπολογισμός της προβολής}




\clearpage
\selectlanguage{english}
\bibliographystyle{unsrt} % Specify bibliography style
\bibliography{references} % Include your bibliography file (references.bib)

\end{document}